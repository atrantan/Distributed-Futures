\chapter*{Introduction}

\paragraph{}
We present an implementation of the future programming model for distributed memory,
using MPI-2's one-sided communication.  The interface is implemented as a runtime 
library that allows the user to  expose parallelism, by issuing callable functor objects 
asynchronously.  The future object is used like a simple communication channel, where the worker process 
will send through it the return value of the functor object. 
A future object is also used for synchronization. Such an object can be accessed at any time during execution.
The process accessing it will block until the worker process finishes the execution of the functor object, and
transmits the result to the future object.

\paragraph{}
Traditionally futures are implemented using threads in shared memory environments.  
In this work we show that the C++11 standard future interface~\cite{CPP:Threads} can be implemented meaningfully 
for distributed memory machines.  We have chosen to build our system using the MPI-2 one-sided
communication library, so that we can explore and evaluate it's potential to provide a completely asynchronous
communication scheme.  Another reason for using an MPI library is that it is the most commonly message passing
library available on distributed and shared memories machines alike.  
The contributions of this work can sum up to:
\\
\begin{itemize}
	\item Implementation of a unified C++ futures interface for both shared and distributed memory machines, as a runtime system.
	\item Performance Evaluation of the our implementation. 
	\item Evaluation of the MPI-2 one-sided communication interface, for implementing an advanced runtime system.
	\item Exploration of the potential of implementing a runtime on distributed memory using shared memory scheduling techniques.
\end{itemize}

\paragraph{}
Our evaluation shows that the interface implementation is possible, but, performance-wise,  our implementation 
is only able to offer some speedup only when we use coarse grain tasks, due to the high cost of issuing functions 
asynchronously and/or inefficient synchronization schemes. Moreover, MPI-2 one-sided communication interface is 
not as versatile as we would like, especially regarding fine grain synchronization.  

\paragraph{}
The rest of this report is organized as follows:  In chapter \ref{chap:background}, we present the current state
and trends in parallel computing.  We briefly introduce the concept of asynchronous execution models and present
the futures programming model.  At the end of the introduction, we describe MPI's one-sided communication interface.
In chapter \ref{chap:related} we discuss other projects related with our work, and present their approach to
asynchronous communication.  In chapter \ref{chap:implementation} we present in details our system's design and 
implementation.  In chapter \ref{chap:evaluation} we assess our efforts in building the C++11 interface using 
MPI's one-sided communication and use some microbenchmarks and real applications to evaluate the performance
of our runtime system.  Finally, in chapter \ref{chap:conclusion} we give our concluding remarks regarding our
library along with our suggestions for its improvement.    

