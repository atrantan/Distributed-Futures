\thispagestyle{empty}

\selectlanguage{greek}

\begin{titlepage}
\begin{center}

{\bf\Large{Περίληψη}}\\

\end{center}

Σε αυτή την εργασία παρουσιάζουμε την υλοποίηση μιας βιβλιοθήκης \textlatin{C++} του προγραμματιστικού μοντέλου
των futures για \textlatin{Distributed Memory} περιβάλλοντα. Η υλοποίησή μας χρησιμοποιεί ένα interface παρόμοιο 
με αυτό της \textlatin{C++ standard library}. Ο χρήστης μπορεί να χρησιμοποιήσει το
\textlatin{interface} των \textlatin{futures} για να εκφράσει παραλληλισμό και να συγχρονίσει την εφαρμογή του, ενώ η
υποκείμενο σύστημα \textlatin{runtime} μας είναι υπεύθυνο για τον καταμερισμό της εργασίας και συνχρονισμό των 
διαφορετικων διεργασιών. Το συστημά μας βασίζεται στην \textlatin{one-sided communication} βιβλιοθήκη του \textlatin{MPI},
για την επίτευξη ασύγχρονης επικοινωνίας. Η αξιολογώντας τις επιδόσεις του \textlatin{runtime} μας, καταλήγουμε 
στο συμπέρασμα ότι, με την τρέχουσα υλοποίηση, είναι κατάλληλο μόνο για το χειρισμό \textlatin{coarse-grain} εργασιών.
Μοιραζόμαστε επίσης την εμπειρία μας χρησιμοποιώντας το \textlatin{MPI one-sided communication interface} για την υλοποίηση 
ενός \textlatin{runtime} συστήματος υψηλών επιδόσεων.

\vfill

\end{titlepage}

\selectlanguage{english}
